%!TEX TS-program = xelatex
%!TEX encoding = UTF-8 Unicode
% Awesome CV LaTeX Template for CV/Resume
%
% This template has been downloaded from:
% https://github.com/posquit0/Awesome-CV
%
% Author:
% Claud D. Park <posquit0.bj@gmail.com>
% http://www.posquit0.com
%
% Template license:
% CC BY-SA 4.0 (https://creativecommons.org/licenses/by-sa/4.0/)
%


%-------------------------------------------------------------------------------
% CONFIGURATIONS
%-------------------------------------------------------------------------------
% A4 paper size by default, use 'letterpaper' for US letter
\documentclass[11pt, a4paper, spanish]{awesome-cv}

\usepackage[utf8]{inputenc}  % For UTF-8 encoding
\usepackage[T1]{fontenc}     % For proper character encoding
\usepackage[spanish]{babel}  % For Spanish language support


% If using XeLaTeX or LuaLaTeX
\usepackage{fontspec}        % For loading system fonts
\defaultfontfeatures{Ligatures=TeX}

% Example of setting the main font to one that supports Spanish characters
\setmainfont{Liberation Serif}

% Configure page margins with geometry
\geometry{left=1.4cm, top=.8cm, right=1.4cm, bottom=1.8cm, footskip=.5cm}

% Color for highlights
% Awesome Colors: awesome-emerald, awesome-skyblue, awesome-red, awesome-pink, awesome-orange
%                 awesome-nephritis, awesome-concrete, awesome-darknight
\colorlet{awesome}{awesome-red}
% Uncomment if you would like to specify your own color
% \definecolor{awesome}{HTML}{3E6D9C}

% Colors for text
% Uncomment if you would like to specify your own color
% \definecolor{darktext}{HTML}{414141}
% \definecolor{text}{HTML}{333333}
% \definecolor{graytext}{HTML}{5D5D5D}
% \definecolor{lighttext}{HTML}{999999}
% \definecolor{sectiondivider}{HTML}{5D5D5D}

% Set false if you don't want to highlight section with awesome color
\setbool{acvSectionColorHighlight}{true}

% If you would like to change the social information separator from a pipe (|) to something else
\renewcommand{\acvHeaderSocialSep}{\quad\textbar\quad}


%-------------------------------------------------------------------------------
%	PERSONAL INFORMATION
%	Comment any of the lines below if they are not required
%-------------------------------------------------------------------------------
% Available options: circle|rectangle,edge/noedge,left/right
% \photo[rectangle,edge,right]{./examples/profile}
\name{Victor}{Chura}
\position{Data Engineer}
\address{Girona, España}

\mobile{(+34) 627-041624}
\email{vmchura@gmail.com}
%\dateofbirth{January 1st, 1970}
%\homepage{www.posquit0.com}
%\github{posquit0}
\linkedin{https://www.linkedin.com/in/victor-chura/}
% \gitlab{gitlab-id}
% \stackoverflow{SO-id}{SO-name}
% \twitter{@twit}
% \skype{skype-id}
% \reddit{reddit-id}
% \medium{madium-id}
% \kaggle{kaggle-id}
% \hackerrank{hackerrank-id}
% \googlescholar{googlescholar-id}{name-to-display}
%% \firstname and \lastname will be used
% \googlescholar{googlescholar-id}{}
% \extrainfo{extra information}

%\quote{``Be the change that you want to see in the world."}


%-------------------------------------------------------------------------------
\begin{document}

% Print the header with above personal information
% Give optional argument to change alignment(C: center, L: left, R: right)
\makecvheader[C]

% Print the footer with 3 arguments(<left>, <center>, <right>)
% Leave any of these blank if they are not needed
\makecvfooter
  {\today}
  {Victor Chura~~~·~~~Curriculum Vitae}
  {\thepage}


%-------------------------------------------------------------------------------
%	CV/RESUME CONTENT
%	Each section is imported separately, open each file in turn to modify content
%-------------------------------------------------------------------------------
%-------------------------------------------------------------------------------
%	SECTION TITLE
%-------------------------------------------------------------------------------
\cvsection{Resumen}


%-------------------------------------------------------------------------------
%	CONTENT
%-------------------------------------------------------------------------------
\begin{cvparagraph}

%---------------------------------------------------------

Analista de Datos en Entelgy.

He participado en proyectos de datos, incluyendo la creaci\'on de procesos ETL, el despliegue y la productivizaci\'on de modelos de Machine Learning.
He participado en m\'as de cuatro proyectos de principio a fin en el sector bancario.
Adem\'as, he desarrollado m\'as de 20 m\'odulos de Big Data utilizando tecnolog\'ias como Spark, Scala y Python.

Desde el inicio de mi carrere participé en proyectos de datos, desarrollé software para procesamiento de imágenes, además de asistentes semiautomatizados para corregir datos GPS y generar ejes de carretera en formato CAD a partir de datos del Sistema de Registro Videográfico Georeferenciado.

Me complace compartir conocimientos con colegas y entusiastas en el campo de Procesamiento de Datos.
En los \'ultimos tres años he realizado capacitaciones, eventualmente desarroll\'e el portal \href{www.miguelacademy.com}{www.miguelacademy.com} .

\end{cvparagraph}

%-------------------------------------------------------------------------------
%	SECTION TITLE
%-------------------------------------------------------------------------------
\cvsection{Experiencia Profesional}


%-------------------------------------------------------------------------------
%	CONTENT
%-------------------------------------------------------------------------------
\begin{cventries}

%---------------------------------------------------------
  \cventry
    {Analista de Datos} % Job title
    {Entelgy Peru} % Organization
    {Lima Peru} % Location
    {2021 - Actualidad} % Date(s)
    {
      \begin{cvitems} % Description(s) of tasks/responsibilities
        \item {Proyecto: Migraci\'on de m\'ultiples procedimientos de SQL a jobs de Apache Spark. Scala / Spark / Python / SQL.}
        \item {Proyecto: Implementaci\'on y despliegue de Modelo de Machine Learning y sus pipelines (m\'ultiples jobs de PySpark) en un entorno de nube que se ejecuta a diario.}
        \item {Proyecto: Implementaci\'on de un modelo de an\'alisis de riesgos. El modelo implementa procesos Spark, se desarrollaron herramientas de IntellIJ IDEA y Jupyter siguiendo todos los est\'andares automatizados como SonarQube, entre otros.}
        \item {Proyecto: Desarrollo e implementaci\'on de M\'odulos de Big Data que se resumen en Dashboards para toma de decisiones.}
      \end{cvitems}
    }

%---------------------------------------------------------
  \cventry
  {Founder} % Job title
  {Abramaris / MiguelAcademy.com} % Organization
  {Lima Peru} % Location
  {2023- Actualidad} % Date(s)
  {
    \begin{cvitems} % Description(s) of tasks/responsibilities
      \item {Desarrollo e implementaci\'on de www.miguelacademy.com}
      \item {Plataforma para la enseñanza de tecnolog\'ias de Big Data y Machine Learning. Spark / Python / SQL.}
    \end{cvitems}
  }
%---------------------------------------------------------
  \cventry
  {Automatizaci\'on de Procesos} % Job title
  {Consorcio Vial Selva Central} % Organization
  {Lima Per\'u} % Location
  {2018-2019} % Date(s)
  {
    \begin{cvitems} % Description(s) of tasks/responsibilities
      \item {Desarrollo y mantenimiento de s}
      \item {Desarrollo y mantenimiento de s}
      \item {Desarrollo de sistema de ingesta de datos de sensores/equipos. Utilizando el lenguaje Scala.}
      \item {Prueba y automatizacion de software para su uso en entornos seguros.}
    \end{cvitems}
  }

%---------------------------------------------------------
  \cventry
    {Automatizaci\'on de Procesos} % Job title
    {Corporaci\'on Mayo} % Organization
    {Lima Per\'u} % Location
    {2016-2018} % Date(s)
    {
      \begin{cvitems} % Description(s) of tasks/responsibilities
        \item {Desarrollo de software de equipo de relevamiento de datos para v\'ias de comunicaci\'on. (Utilizando Machine Learning y C# .NET)}
        \item {Prueba y automatizaci\'on de software para ser utilizado en entornos seguros.}
        \item {Inspecci\'on y ensamblaje de componentes electr\'onicos para uso seguro y eficiente.}
        \item {Mantuve el seguimiento de los plazos para los tiempos del ciclo de desarrollo.}

      \end{cvitems}
    }

%---------------------------------------------------------
\end{cventries}

%\input{resume/honors.tex}
%%-------------------------------------------------------------------------------
%	SECTION TITLE
%-------------------------------------------------------------------------------
\cvsection{Certificados}


%-------------------------------------------------------------------------------
%	CONTENT
%-------------------------------------------------------------------------------
\begin{cvhonors}
%---------------------------------------------------------
  \cvhonor
  {Functional Programming in Scala} % Name
  {École Polytechnique Fédérale de Lausanne} % Issuer
  {QPYSGGXBEAYA} % Credential ID
  {2018} % Date(s)

%---------------------------------------------------------
  \cvhonor
  {DevOps on AWS Specialization} % Name
  {Amazon Web Services} % Issuer
  {MTXGTSHK7TNB} % Credential ID
  {2023} % Date(s)

%---------------------------------------------------------
  \cvhonor
  {DevOps Leader Certification} % Name
  {Centro Internacional de Liderazgo / Liderazgo Exponencial} % Issuer
  {} % Credential ID
  {2023} % Date(s)

%---------------------------------------------------------
  \cvhonor
  {Scrum Master} % Name
  {European Scrum} % Issuer
  {} % Credential ID
  {2022} % Date(s)
\end{cvhonors}

% \input{resume/presentation.tex}
% \input{resume/writing.tex}
% \input{resume/committees.tex}
%-------------------------------------------------------------------------------
%	SECTION TITLE
%-------------------------------------------------------------------------------
\cvsection{Education}


%-------------------------------------------------------------------------------
%	CONTENT
%-------------------------------------------------------------------------------
\begin{cventries}

%---------------------------------------------------------
  \cventry
    {Bachiller Ingeniería Mecatrónica} % Degree
    {Univerisdad Nacional de Ingeniería} % Institution
    {Lima - Perú} % Location
    {2009 - 2014} % Date(s)
    {
      \begin{cvitems} % Description(s) bullet points
        \item {Quito superior en la Facultad de Ingeniería Mecánica.}
      \end{cvitems}
    }

%---------------------------------------------------------
\end{cventries}

% %-------------------------------------------------------------------------------
%	SECTION TITLE
%-------------------------------------------------------------------------------
\cvsection{Repositorio público}


%-------------------------------------------------------------------------------
%	CONTENT
%-------------------------------------------------------------------------------
\begin{cventries}


%---------------------------------------------------------

  \cventry
      {Desarrollador} % Affiliation/role
      {https://github.com/vmchura/vevial/tree/master} % Organization/group
      {rocesamiento de datos geoposicionamiento} % Location
      {2022 - Actualidad} % Date(s)
    {
      \begin{cvitems} % Description(s) of experience/contributions/knowledge
        \item {Desarrollo de respositorio público de procesamiento de datos georeferenciados a lo largo del eje de una carretera. La mayor parte de procesamiento es en lenguaje Scala. Se leen datos de GPS de varios segmentos de carretera, el resultado es un geoposicionamiento de las imágenes tomadas por cámaras externas eliminando errores de lectura o ausencia de GPS.}
      \end{cvitems}
    }
%---------------------------------------------------------
\end{cventries}



%-------------------------------------------------------------------------------
\end{document}
